\documentclass[final,hyperref={pdfpagelabels=false}]{beamer}
\usepackage[orientation=portrait,size=a0,scale=1.7]{beamerposter}
\usetheme{I6pd2} % Use the I6pd2 theme supplied with this template

%%%%%%%%%%%%%%%%%%%%%%%%%%%%%%%%%%%%%%%%%%
%% Packages
%%%%%%%%%%%%%%%%%%%%%%%%%%%%%%%%%%%%%%%%%%

% Tikz
\usepackage{tikz}
\usetikzlibrary{chains,arrows,automata,fit,positioning,calc}

% Colors
\usepackage{xcolor}

% Images
\usepackage{graphics}
\graphicspath{{img/}}

% Math Symbols
\usepackage{amsmath,amsthm,amssymb,latexsym,stmaryrd}

%%%%%%%%%%%%%%%%%%%%%%%%%%%%%%%%%%%%%%%%%%
%% Macros
%%%%%%%%%%%%%%%%%%%%%%%%%%%%%%%%%%%%%%%%%%
\renewcommand\alert[1]{\textcolor{mLightBrown}{#1}}

%%%%%%%%%%%%%%%%%%%%%%%%%%%%%%%%%%%%%%%%%%
%% Agda imports
%%%%%%%%%%%%%%%%%%%%%%%%%%%%%%%%%%%%%%%%%%
%% ODER: format ==         = "\mathrel{==}"
%% ODER: format /=         = "\neq "
%
%
\makeatletter
\@ifundefined{lhs2tex.lhs2tex.sty.read}%
  {\@namedef{lhs2tex.lhs2tex.sty.read}{}%
   \newcommand\SkipToFmtEnd{}%
   \newcommand\EndFmtInput{}%
   \long\def\SkipToFmtEnd#1\EndFmtInput{}%
  }\SkipToFmtEnd

\newcommand\ReadOnlyOnce[1]{\@ifundefined{#1}{\@namedef{#1}{}}\SkipToFmtEnd}
\usepackage{amstext}
\usepackage{amssymb}
\usepackage{stmaryrd}
\DeclareFontFamily{OT1}{cmtex}{}
\DeclareFontShape{OT1}{cmtex}{m}{n}
  {<5><6><7><8>cmtex8
   <9>cmtex9
   <10><10.95><12><14.4><17.28><20.74><24.88>cmtex10}{}
\DeclareFontShape{OT1}{cmtex}{m}{it}
  {<-> ssub * cmtt/m/it}{}
\newcommand{\texfamily}{\fontfamily{cmtex}\selectfont}
\DeclareFontShape{OT1}{cmtt}{bx}{n}
  {<5><6><7><8>cmtt8
   <9>cmbtt9
   <10><10.95><12><14.4><17.28><20.74><24.88>cmbtt10}{}
\DeclareFontShape{OT1}{cmtex}{bx}{n}
  {<-> ssub * cmtt/bx/n}{}
\newcommand{\tex}[1]{\text{\texfamily#1}}	% NEU

\newcommand{\Sp}{\hskip.33334em\relax}


\newcommand{\Conid}[1]{\mathit{#1}}
\newcommand{\Varid}[1]{\mathit{#1}}
\newcommand{\anonymous}{\kern0.06em \vbox{\hrule\@width.5em}}
\newcommand{\plus}{\mathbin{+\!\!\!+}}
\newcommand{\bind}{\mathbin{>\!\!\!>\mkern-6.7mu=}}
\newcommand{\rbind}{\mathbin{=\mkern-6.7mu<\!\!\!<}}% suggested by Neil Mitchell
\newcommand{\sequ}{\mathbin{>\!\!\!>}}
\renewcommand{\leq}{\leqslant}
\renewcommand{\geq}{\geqslant}
\usepackage{polytable}

%mathindent has to be defined
\@ifundefined{mathindent}%
  {\newdimen\mathindent\mathindent\leftmargini}%
  {}%

\def\resethooks{%
  \global\let\SaveRestoreHook\empty
  \global\let\ColumnHook\empty}
\newcommand*{\savecolumns}[1][default]%
  {\g@addto@macro\SaveRestoreHook{\savecolumns[#1]}}
\newcommand*{\restorecolumns}[1][default]%
  {\g@addto@macro\SaveRestoreHook{\restorecolumns[#1]}}
\newcommand*{\aligncolumn}[2]%
  {\g@addto@macro\ColumnHook{\column{#1}{#2}}}

\resethooks

\newcommand{\onelinecommentchars}{\quad-{}- }
\newcommand{\commentbeginchars}{\enskip\{-}
\newcommand{\commentendchars}{-\}\enskip}

\newcommand{\visiblecomments}{%
  \let\onelinecomment=\onelinecommentchars
  \let\commentbegin=\commentbeginchars
  \let\commentend=\commentendchars}

\newcommand{\invisiblecomments}{%
  \let\onelinecomment=\empty
  \let\commentbegin=\empty
  \let\commentend=\empty}

\visiblecomments

\newlength{\blanklineskip}
\setlength{\blanklineskip}{0.66084ex}

\newcommand{\hsindent}[1]{\quad}% default is fixed indentation
\let\hspre\empty
\let\hspost\empty
\newcommand{\NB}{\textbf{NB}}
\newcommand{\Todo}[1]{$\langle$\textbf{To do:}~#1$\rangle$}

\EndFmtInput
\makeatother
%
%
%
%
%
%
% This package provides two environments suitable to take the place
% of hscode, called "plainhscode" and "arrayhscode". 
%
% The plain environment surrounds each code block by vertical space,
% and it uses \abovedisplayskip and \belowdisplayskip to get spacing
% similar to formulas. Note that if these dimensions are changed,
% the spacing around displayed math formulas changes as well.
% All code is indented using \leftskip.
%
% Changed 19.08.2004 to reflect changes in colorcode. Should work with
% CodeGroup.sty.
%
\ReadOnlyOnce{polycode.fmt}%
\makeatletter

\newcommand{\hsnewpar}[1]%
  {{\parskip=0pt\parindent=0pt\par\vskip #1\noindent}}

% can be used, for instance, to redefine the code size, by setting the
% command to \small or something alike
\newcommand{\hscodestyle}{}

% The command \sethscode can be used to switch the code formatting
% behaviour by mapping the hscode environment in the subst directive
% to a new LaTeX environment.

\newcommand{\sethscode}[1]%
  {\expandafter\let\expandafter\hscode\csname #1\endcsname
   \expandafter\let\expandafter\endhscode\csname end#1\endcsname}

% "compatibility" mode restores the non-polycode.fmt layout.

\newenvironment{compathscode}%
  {\par\noindent
   \advance\leftskip\mathindent
   \hscodestyle
   \let\\=\@normalcr
   \let\hspre\(\let\hspost\)%
   \pboxed}%
  {\endpboxed\)%
   \par\noindent
   \ignorespacesafterend}

\newcommand{\compaths}{\sethscode{compathscode}}

% "plain" mode is the proposed default.
% It should now work with \centering.
% This required some changes. The old version
% is still available for reference as oldplainhscode.

\newenvironment{plainhscode}%
  {\hsnewpar\abovedisplayskip
   \advance\leftskip\mathindent
   \hscodestyle
   \let\hspre\(\let\hspost\)%
   \pboxed}%
  {\endpboxed%
   \hsnewpar\belowdisplayskip
   \ignorespacesafterend}

\newenvironment{oldplainhscode}%
  {\hsnewpar\abovedisplayskip
   \advance\leftskip\mathindent
   \hscodestyle
   \let\\=\@normalcr
   \(\pboxed}%
  {\endpboxed\)%
   \hsnewpar\belowdisplayskip
   \ignorespacesafterend}

% Here, we make plainhscode the default environment.

\newcommand{\plainhs}{\sethscode{plainhscode}}
\newcommand{\oldplainhs}{\sethscode{oldplainhscode}}
\plainhs

% The arrayhscode is like plain, but makes use of polytable's
% parray environment which disallows page breaks in code blocks.

\newenvironment{arrayhscode}%
  {\hsnewpar\abovedisplayskip
   \advance\leftskip\mathindent
   \hscodestyle
   \let\\=\@normalcr
   \(\parray}%
  {\endparray\)%
   \hsnewpar\belowdisplayskip
   \ignorespacesafterend}

\newcommand{\arrayhs}{\sethscode{arrayhscode}}

% The mathhscode environment also makes use of polytable's parray 
% environment. It is supposed to be used only inside math mode 
% (I used it to typeset the type rules in my thesis).

\newenvironment{mathhscode}%
  {\parray}{\endparray}

\newcommand{\mathhs}{\sethscode{mathhscode}}

% texths is similar to mathhs, but works in text mode.

\newenvironment{texthscode}%
  {\(\parray}{\endparray\)}

\newcommand{\texths}{\sethscode{texthscode}}

% The framed environment places code in a framed box.

\def\codeframewidth{\arrayrulewidth}
\RequirePackage{calc}

\newenvironment{framedhscode}%
  {\parskip=\abovedisplayskip\par\noindent
   \hscodestyle
   \arrayrulewidth=\codeframewidth
   \tabular{@{}|p{\linewidth-2\arraycolsep-2\arrayrulewidth-2pt}|@{}}%
   \hline\framedhslinecorrect\\{-1.5ex}%
   \let\endoflinesave=\\
   \let\\=\@normalcr
   \(\pboxed}%
  {\endpboxed\)%
   \framedhslinecorrect\endoflinesave{.5ex}\hline
   \endtabular
   \parskip=\belowdisplayskip\par\noindent
   \ignorespacesafterend}

\newcommand{\framedhslinecorrect}[2]%
  {#1[#2]}

\newcommand{\framedhs}{\sethscode{framedhscode}}

% The inlinehscode environment is an experimental environment
% that can be used to typeset displayed code inline.

\newenvironment{inlinehscode}%
  {\(\def\column##1##2{}%
   \let\>\undefined\let\<\undefined\let\\\undefined
   \newcommand\>[1][]{}\newcommand\<[1][]{}\newcommand\\[1][]{}%
   \def\fromto##1##2##3{##3}%
   \def\nextline{}}{\) }%

\newcommand{\inlinehs}{\sethscode{inlinehscode}}

% The joincode environment is a separate environment that
% can be used to surround and thereby connect multiple code
% blocks.

\newenvironment{joincode}%
  {\let\orighscode=\hscode
   \let\origendhscode=\endhscode
   \def\endhscode{\def\hscode{\endgroup\def\@currenvir{hscode}\\}\begingroup}
   %\let\SaveRestoreHook=\empty
   %\let\ColumnHook=\empty
   %\let\resethooks=\empty
   \orighscode\def\hscode{\endgroup\def\@currenvir{hscode}}}%
  {\origendhscode
   \global\let\hscode=\orighscode
   \global\let\endhscode=\origendhscode}%

\makeatother
\EndFmtInput
%
%%%%%%%%%%%%%%%%%%%%%%%%%%%%%%
%% Agda Styling

\newenvironment{agda}
{}
{}

% Bitcoin symbol
\def\bitcoin{%
  \leavevmode
  \vtop{\offinterlineskip %\bfseries
    \setbox0=\hbox{B}%
    \setbox2=\hbox to\wd0{\hfil\hskip-.03em
    \vrule height .6ex width .15ex\hskip .08em
    \vrule height .6ex width .15ex\hfil}
    \vbox{\copy2\box0}\box2}}

% Text superscript
\newcommand\textsup[1]{\!\text{\textsuperscript{#1}}}
\newcommand\textsub[1]{\!\text{\textsubscript{#1}}}

% Horizontal lines for inference rules
\newcommand\inferLine[1]{\rule[3pt]{#1}{.6pt}}
\newcommand\inferSmall{\inferLine{2cm}}
\newcommand\inferMedium{\inferLine{5cm}}
\newcommand\inferLarge{\inferLine{8cm}}
\newcommand\inferVeryLarge{\inferLine{11cm}}

%% Colors (from duo-tone light syntax)
\definecolor{hsblack}{RGB}{45,32,3}
\definecolor{hsgold1}{RGB}{179,169,149}
\definecolor{hsgold2}{RGB}{177,149,90}
\definecolor{hsgold3}{RGB}{190,106,13}%{192,96,4}%{132,97,19}
\definecolor{hsblue1}{RGB}{173,176,182}
\definecolor{hsblue2}{RGB}{113,142,205}
\definecolor{hsblue3}{RGB}{0,33,132}
\definecolor{hsblue4}{RGB}{97,108,132}
\definecolor{hsblue5}{RGB}{34,50,68}
\definecolor{hsred2}{RGB}{191,121,103}
\definecolor{hsred3}{RGB}{171,72,46}

%% LaTeX Kerning nastiness. By using curly braces to delimit color group,
%% it breaks spacing. The following seems to work:
%%
%% https://tex.stackexchange.com/questions/85033/colored-symbols/85035#85035
%%
\newcommand*{\mathcolor}{}
\def\mathcolor#1#{\mathcoloraux{#1}}
\newcommand*{\mathcoloraux}[3]{%
  \protect\leavevmode
  \begingroup
    \color#1{#2}#3%
  \endgroup
}

\newcommand{\HSKeyword}[1]{\mathcolor{hsgold3}{\textbf{#1}}}
\newcommand{\HSNumeral}[1]{\mathcolor{hsred3}{#1}}
\newcommand{\HSChar}[1]{\mathcolor{hsred2}{#1}}
\newcommand{\HSString}[1]{\mathcolor{hsred2}{#1}}
\newcommand{\HSSpecial}[1]{\mathcolor{hsblue4}{\ensuremath{#1}}}
\newcommand{\HSSym}[1]{\mathcolor{hsblue4}{\ensuremath{#1}}}
\newcommand{\HSCon}[1]{\mathcolor{hsblue3}{#1}}
\newcommand{\HSVar}[1]{\mathcolor{hsblue5}{\mathit{\ensuremath{#1}}}}
\newcommand{\HSComment}[1]{\mathcolor{hsgold2}{\textit{#1}}}


%% Keywords


%% Constructors


%%% Formatting


%%format λ       = "\HSSym{\lambda\ }"
%%format →       = "\HSSym{\mathrel{\rightarrow}}"





\def\commentbegin{}
\def\commentend{}


%%%%%%%%%%%%%%%%%%%%%%%%%%%%%%%%%%%%%%%%%%
%% Fonts
%%%%%%%%%%%%%%%%%%%%%%%%%%%%%%%%%%%%%%%%%%
\usepackage{relsize}
\usepackage[tt=false]{libertine}
\usepackage[libertine]{newtxmath}

%----------------------------------------------------------------------------------------
% TITLE SECTION
%----------------------------------------------------------------------------------------

\title{\LARGE \textsc{Formalizing BitML Calculus in Agda}}
\author{Orestis Melkonian}
\date{July 8, 2019}
\institute{Utrecht University, The Netherlands}

\titlegraphic{
\vspace{3.8cm}\flushright\includegraphics[scale=.25]{uu}
\vspace{.3cm}\flushright\includegraphics[scale=.375]{iohk}\hspace{.1cm}
}

%----------------------------------------------------------------------------------------
% FOOTER TEXT
%----------------------------------------------------------------------------------------

\newcommand\footsize{15ex}
\newcommand\leftfoot{
  \begin{minipage}{.45\textwidth}
  \topskip0pt
  \vspace{\fill}
  \includegraphics[scale=1]{uu}
  \vspace{\fill}
  \end{minipage}
} % Left footer text
\newcommand\rightfoot{
  \begin{minipage}{.55\textwidth}
  \topskip0pt
  \vspace{\fill}
  \large \textbf{https://github.com/omelkonian/formal-bitml}
  \vspace{\fill}
  \end{minipage}
} % Right footer text

%----------------------------------------------------------------------------------------
% SIZES ( 3*sepsize + 2*colsize == 1 )
%----------------------------------------------------------------------------------------
\newcommand\sepsize{.05\textwidth}
\newcommand\colsize{.425\textwidth}

\begin{document}

\addtobeamertemplate{block end}{}{\vspace*{4ex}} % White space under blocks

\begin{frame}[t] % The whole poster is enclosed in one beamer frame

\begin{columns}[t] % The whole poster consists of two major columns, each of which can be subdivided further with another \begin{columns} block - the [t] argument aligns each column's content to the top

\begin{column}{\sepsize}\end{column} % Empty spacer column

\begin{column}{\colsize} % The first column

%----------------------------------------------------------------------------------------
% CONTENT
%----------------------------------------------------------------------------------------

\begin{block}{Introduction}
  \begin{itemize}
    \item \textbf{Janus} is an imperative reversible programming language, meaning that every computation and function can be reversed.
    \vspace{1cm}
    \item \textbf{Hanus} is an extended implementation of Janus that can be compiled straight to Haskell. Because of this, Hanus contains many awesome Haskell features that are unthinkable in regular Janus!
  \end{itemize}
\end{block}

\begin{block}{Reverse your program}
\begin{agda}\begin{hscode}\SaveRestoreHook
\column{B}{@{}>{\hspre}l<{\hspost}@{}}%
\column{3}{@{}>{\hspre}l<{\hspost}@{}}%
\column{5}{@{}>{\hspre}c<{\hspost}@{}}%
\column{5E}{@{}l@{}}%
\column{9}{@{}>{\hspre}l<{\hspost}@{}}%
\column{17}{@{}>{\hspre}c<{\hspost}@{}}%
\column{17E}{@{}l@{}}%
\column{20}{@{}>{\hspre}l<{\hspost}@{}}%
\column{66}{@{}>{\hspre}l<{\hspost}@{}}%
\column{70}{@{}>{\hspre}c<{\hspost}@{}}%
\column{70E}{@{}l@{}}%
\column{74}{@{}>{\hspre}l<{\hspost}@{}}%
\column{E}{@{}>{\hspre}l<{\hspost}@{}}%
\>[B]{}\HSKeyword{data}\;\HSSym{\anonymous}\HSSym{\mathrel{\longrightarrow}}\HSSym{\anonymous}\mathbin{:}\HSCon{Configuration}\;\HSVar{ads}\;\HSVar{cs}\;\HSVar{ds}\HSSym{\mathbin{\to}}\HSCon{Configuration}\;\HSVar{ads}\hspace{2pt}^{\mathsmaller{\prime}}\HSVar{cs}\hspace{2pt}^{\mathsmaller{\prime}}\HSVar{ds}\hspace{2pt}^{\mathsmaller{\prime}}\HSSym{\mathbin{\to}}\HSCon{Set}\;\HSKeyword{where}{}\<[E]%
\\[\blanklineskip]%
\>[B]{}\hsindent{3}{}\<[3]%
\>[3]{}\HSCon{D}\HSSym{\text{\textit{-}}}\HSCon{AuthJoin}\mathbin{:}{}\<[66]%
\>[66]{}\HSCon{D}\HSSym{\text{\textit{-}}}\HSCon{Join}\mathbin{:}{}\<[E]%
\\
\>[3]{}\hsindent{6}{}\<[9]%
\>[9]{}\HSSym{\langle\ }\HSCon{A}\HSSpecial{\HSSym{\mathbin{,}}}\HSVar{v}\HSSym{\ \rangle}\HSSym{\textsup{d}}\HSSym{\mathbin{\mid}}\HSSym{\langle\ }\HSCon{A}\HSSpecial{\HSSym{\mathbin{,}}}\HSVar{v}\hspace{2pt}^{\mathsmaller{\prime}}\HSSym{\ \rangle}\HSSym{\textsup{d}}\HSSym{\mathbin{\mid}}\HSSym{\Gamma}{}\<[74]%
\>[74]{}\HSSym{\langle\ }\HSCon{A}\HSSpecial{\HSSym{\mathbin{,}}}\HSVar{v}\HSSym{\ \rangle}\HSSym{\textsup{d}}\HSSym{\mathbin{\mid}}\HSSym{\langle\ }\HSCon{A}\HSSpecial{\HSSym{\mathbin{,}}}\HSVar{v}\hspace{2pt}^{\mathsmaller{\prime}}\HSSym{\ \rangle}\HSSym{\textsup{d}}\HSSym{\mathbin{\mid}}\HSCon{A}\;\HSSpecial{[\mskip1.5mu }\HSNumeral{0}\HSSym{\mathrel{\leftrightarrow}}\HSNumeral{1}\HSSpecial{\mskip1.5mu]}\HSSym{\mathbin{\mid}}\HSSym{\Gamma}{}\<[E]%
\\
\>[3]{}\hsindent{2}{}\<[5]%
\>[5]{}\HSSym{\mathrel{\longrightarrow}}{}\<[5E]%
\>[9]{}\HSSym{\langle\ }\HSCon{A}\HSSpecial{\HSSym{\mathbin{,}}}\HSVar{v}\HSSym{\ \rangle}\HSSym{\textsup{d}}\HSSym{\mathbin{\mid}}\HSSym{\langle\ }\HSCon{A}\HSSpecial{\HSSym{\mathbin{,}}}\HSVar{v}\hspace{2pt}^{\mathsmaller{\prime}}\HSSym{\ \rangle}\HSSym{\textsup{d}}\HSSym{\mathbin{\mid}}\HSCon{A}\;\HSSpecial{[\mskip1.5mu }\HSNumeral{0}\HSSym{\mathrel{\leftrightarrow}}\HSNumeral{1}\HSSpecial{\mskip1.5mu]}\HSSym{\mathbin{\mid}}\HSSym{\Gamma}\;\HSSym{\textsup{p}}\;\HSSym{\textsup{p}}\;\HSSym{\textsup{p}}\;{}\<[66]%
\>[66]{}\HSSym{\textsup{p}}{}\<[70]%
\>[70]{}\HSSym{\mathrel{\longrightarrow}}{}\<[70E]%
\>[74]{}\HSSym{\langle\ }\HSCon{A}\HSSpecial{\HSSym{\mathbin{,}}}\HSVar{v}\HSSym{+}\HSVar{v}\hspace{2pt}^{\mathsmaller{\prime}}\HSSym{\ \rangle}\HSSym{\textsup{d}}\HSSym{\mathbin{\mid}}\HSSym{\Gamma}{}\<[E]%
\\[\blanklineskip]%
\>[B]{}\hsindent{3}{}\<[3]%
\>[3]{}\HSCon{C}\HSSym{\text{\textit{-}}}\HSCon{Advertise}{}\<[17]%
\>[17]{}\mathbin{:}{}\<[17E]%
\>[20]{}\HSCon{Any}\;\HSSpecial{(}\HSSym{\_\!}\HSSym{\mathrel{\in}}\HSCon{Hon}\HSSpecial{)}\;\HSSpecial{(}\HSVar{participants}\;\HSSpecial{(}\HSCon{G}\;\HSVar{ad}\HSSpecial{)}\HSSpecial{)}\HSSym{\mathbin{\to}}\HSSpecial{(}\HSSym{\Gamma}\HSSym{\mathrel{\longrightarrow}}\HSVar{ad}\HSSym{\mathbin{\mid}}\HSSym{\Gamma}\HSSpecial{)}{}\<[E]%
\\[\blanklineskip]%
\>[B]{}\hsindent{3}{}\<[3]%
\>[3]{}\HSCon{C}\HSSym{\text{\textit{-}}}\HSCon{AuthCommit}{}\<[17]%
\>[17]{}\mathbin{:}{}\<[17E]%
\>[20]{}\HSSpecial{(}\HSVar{secrets}\;\HSCon{A}\;\HSSpecial{(}\HSCon{G}\;\HSVar{ad}\HSSpecial{)}\HSSym{\mathrel{\equiv}}\HSVar{a₀}\;\HSSym{\dots}\;\HSVar{aₙ}\HSSpecial{)}\HSSym{\mathrel{\times}}\HSSpecial{(}\HSCon{A}\HSSym{\mathrel{\in}}\HSCon{Hon}\HSSym{\mathbin{\to}}\HSCon{All}\;\HSSpecial{(}\HSSym{\_}\HSSym{\mathrel{\not\equiv}}\;\HSVar{nothing}\HSSpecial{)}\;\HSVar{a}\;\HSSym{\textsub{i}}\HSSpecial{)}{}\<[E]%
\\
\>[17]{}\HSSym{\mathbin{\to}}{}\<[17E]%
\>[20]{}\HSSpecial{`}\HSVar{ad}\HSSym{\mathbin{\mid}}\HSSym{\Gamma}\HSSym{\mathrel{\longrightarrow}}\HSSpecial{`}\HSVar{ad}\HSSym{\mathbin{\mid}}\HSSym{\Gamma}\HSSym{\mathbin{\mid}}\HSSym{\dots}\HSSym{\langle\ }\HSCon{A}\mathbin{:}\HSVar{a}\;\HSSym{\textsub{i}}\HSSym{\hspace{1pt}\#}\HSCon{N}\;\HSSym{\textsub{i}}\HSSym{\ \rangle}\HSSym{\dots}\;\HSSym{\! | \!}\;\HSCon{A}\;\HSSpecial{[\mskip1.5mu }\HSSym{♯▷}\;\HSVar{ad}\HSSpecial{\mskip1.5mu]}{}\<[E]%
\ColumnHook
\end{hscode}\resethooks
\end{agda}
\end{block}

%----------------------------------------------------------------------------------------

\end{column} % End of the first column

\begin{column}{\sepsize}\end{column} % Empty spacer column

\begin{column}{\colsize} % The second column
\addtobeamertemplate{block end}{}{\vspace*{-2.4ex}} % White space under blocks

%----------------------------------------------------------------------------------------

\begin{block}{Semantic Checking (Janus side)}
\end{block}

\begin{block}{Semantic Checking (Haskell side)}
\end{block}

\begin{block}{Haskell Power}
\end{block}

%----------------------------------------------------------------------------------------

\end{column} % End of the second column

\begin{column}{\sepsize}\end{column} % Empty spacer column

\end{columns} % End of all the columns in the poster

\end{frame} % End of the enclosing frame

\end{document}
