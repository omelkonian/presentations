%% \documentclass[aspectratio=43]{beamer}
\documentclass[aspectratio=169]{beamer}
\usetheme[
  block=fill,
  background=light,
  titleformat=smallcaps,
  progressbar=frametitle,
  numbering=none,
]{metropolis}
%% \setbeamersize{text margin left=.5cm,text margin right=.5cm}
\usepackage{appendixnumberbeamer}

%%%%%%%%%%%%%%%%%%%%%%%%%%%%%%%%%%%%%%%%%%
%% Packages
%%%%%%%%%%%%%%%%%%%%%%%%%%%%%%%%%%%%%%%%%%

\usepackage{subfiles} % for modular TeX development
\usepackage{xargs} % multiple optional arguments
\usepackage{xspace} % no-argument aliases
\usepackage{environ} % flexible environments

% Tikz
\usepackage{tikz}
\usetikzlibrary{arrows,positioning,matrix,fit,backgrounds,decorations.text,decorations.pathmorphing,calc,shapes}

% Colors
\usepackage{xcolor}
\usepackage{contour}

% Images
\usepackage{graphics}
\graphicspath{{img/}}

% Quotes
\usepackage{csquotes}
\SetBlockThreshold{2}
\newcommand\myblockquote[2]{%
  \blockquote{\hspace*{2em}\emph{`#1'}#2}\par}

% Agda
\usepackage{agda}
\usepackage{fontspec}
\newcommand\agdaFont{%
JuliaMono%
%% DejaVuSansMono%
%% mononoki%
%% Linux Libertine O%
}
\setlength{\mathindent}{0em}
%% \AtBeginEnvironment{code}{\fontsize{9}{12}\selectfont}
\newfontfamily{\AgdaSerifFont}{\agdaFont}
\newfontfamily{\AgdaSansSerifFont}{\agdaFont}
\newfontfamily{\AgdaTypewriterFont}{\agdaFont}
\renewcommand{\AgdaFontStyle}[1]{{\AgdaSansSerifFont{}#1}}
\renewcommand{\AgdaKeywordFontStyle}[1]{{\AgdaSansSerifFont{}#1}}
\renewcommand{\AgdaStringFontStyle}[1]{{\AgdaTypewriterFont{}#1}}
\renewcommand{\AgdaCommentFontStyle}[1]{{\AgdaTypewriterFont{}#1}}
\renewcommand{\AgdaBoundFontStyle}[1]{\textit{\AgdaSerifFont{}#1}}

%% \usepackage{amsxtra}
%% \usepackage{newunicodechar}
%% \newunicodechar{∷}{::}

%%%%%%%%%%%%%%%%%%%%%%%%%%%%%%%%%%%%%%%%%%
%% Fonts
%%%%%%%%%%%%%%%%%%%%%%%%%%%%%%%%%%%%%%%%%%
%% \usepackage{relsize}
%% \usepackage[tt=false]{libertine}
%% \usepackage{yfonts}

%%%%%%%%%%%%%%%%%%%%%%%%%%%%%%%%%%%%%%%%%%
%% Macros
%%%%%%%%%%%%%%%%%%%%%%%%%%%%%%%%%%%%%%%%%%
\renewcommand\alert[1]{\textcolor{mLightBrown}{#1}}
\newcommand\todo[1]{\textcolor{red}{#1}}

\title{Nominal techniques as an Agda library}

\author{Murdoch J. Gabbay, \alert{Orestis Melkonian}}
\date{20 June 2023, ProgLog seminar @ Chalmers}

\begin{document}

% ** Agda setup
\AgdaNoSpaceAroundCode{}

\setbeamerfont{title}{size=\LARGE}
\setbeamerfont{subtitle}{size=\Large}

%% \titlegraphic{}

\begin{center}
\maketitle
\end{center}

\begin{frame}{\alert{No need} for special language support?}
\myblockquote{\textit{
While nominal techniques provide a general approach to define and reason about syntax with binders, I actually hesitate to include it here because it seems to require special language support to use effectively, which Agda does not have.
}}{\\---Jesper Cockx's blog: 1001 Representations of Syntax with Binding}
\end{frame}

\begin{frame}{Motivation}
\begin{itemize}
\item Explore another \alert{point in the design space} of already existing nominal implementations (e.g. Nominal Isabelle)
\item Provide a \alert{constructive} perspective on nominal techniques
\item Do this \alert{without changing the system itself} --- as an Agda library
\item Make it \alert{ergonomic} for the user to use the library as a tool for dealing with names
(e.g. working on some syntax with binding)
\item Mechanise existing (but also new?) \alert{meta-theoretical results}
\end{itemize}
\end{frame}

\section{The nominal universe}
\subfile{Nominal/Swap/Base.tex}
\subfile{Nominal/Abs/Base.tex}
\subfile{Nominal/New.tex}
\subfile{Nominal/Support.tex}
\subfile{Nominal/Abs/Support.tex}

\section{Case study: the untyped $\lambda$-calculus}
\subfile{ULC/Base.tex}
\subfile{ULC/Alpha.tex}
\subfile{ULC/Substitution.tex}
\subfile{ULC/Reduction.tex}

\begin{frame}{Future Work}
\begin{itemize}
\item More meta-programming automation to minimise overhead
  \begin{itemize}
  \item corresponding laws and equivariance lemmas follow the same type-directed structure
 as the swap operation itself
  \end{itemize}
\item Another case study on \emph{cut elimination} for first-order logic
  \begin{itemize}
  \item need to work with entities that are not finitely supported
  \item also includes name abstraction over proof trees
  \end{itemize}
\item Formalise the constructive \emph{total} concretion function, which seems novel
\end{itemize}
\end{frame}

\begin{frame}[standout]
Questions?
\vfill
\begin{center}
\alert{\url{https://omelkonian.github.io/nominal-agda}}
\end{center}
\end{frame}

\end{document}
