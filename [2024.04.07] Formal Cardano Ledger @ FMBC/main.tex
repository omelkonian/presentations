%% \documentclass[aspectratio=43]{beamer}
\documentclass[aspectratio=169]{beamer}
\usetheme[
  block=fill,
  background=light,
  titleformat=smallcaps,
  progressbar=frametitle,
  numbering=none,
]{metropolis}
%% \setbeamersize{text margin left=.5cm,text margin right=.5cm}
\usepackage{appendixnumberbeamer}

%%%%%%%%%%%%%%%%%%%%%%%%%%%%%%%%%%%%%%%%%%
%% Packages
%%%%%%%%%%%%%%%%%%%%%%%%%%%%%%%%%%%%%%%%%%

%% \usepackage{subfiles} % for modular TeX development
\usepackage{xargs} % multiple optional arguments
\usepackage{xspace} % no-argument aliases
\usepackage{environ} % flexible environments

% Tikz
\usepackage{tikz}
\usetikzlibrary{arrows,positioning,matrix,fit,backgrounds,decorations.text,decorations.pathmorphing,calc,shapes}

% Colors
\usepackage{xcolor}
\usepackage{contour}

% Images
\usepackage{graphics}
\graphicspath{{img/}}

% Quotes
\usepackage{csquotes}
\SetBlockThreshold{2}
\newcommand\myblockquote[2]{%
  \blockquote{\hspace*{2em}\emph{`#1'}#2}\par}

% Agda
\usepackage{agda}
%% \renewcommand\\[1][XXX]{\newline} % fix `\\[\AgdaEmptyExtraSkip]%` occurrences
\usepackage{fontspec}
\newcommand\agdaFont{StrippedJuliaMono}
\newcommand\agdaFontOptions{
Path=fonts/,
Extension=.ttf,
UprightFont=*-Regular,
BoldFont=*-Bold,
ItalicFont=*-RegularItalic,
BoldItalicFont=*-BoldItalic,
%% Scale=MatchLowercase
Scale=0.80
%% Scale=MatchUppercase
}
\newfontfamily{\AgdaSerifFont}{\agdaFont}[\agdaFontOptions]
\newfontfamily{\AgdaSansSerifFont}{\agdaFont}[\agdaFontOptions]
\newfontfamily{\AgdaTypewriterFont}{\agdaFont}[\agdaFontOptions]
\renewcommand{\AgdaFontStyle}[1]{{\AgdaSansSerifFont{}#1}}
\renewcommand{\AgdaKeywordFontStyle}[1]{{\AgdaSansSerifFont{}#1}}
\renewcommand{\AgdaStringFontStyle}[1]{{\AgdaTypewriterFont{}#1}}
\renewcommand{\AgdaCommentFontStyle}[1]{{\AgdaTypewriterFont{}#1}}
\renewcommand{\AgdaBoundFontStyle}[1]{{\emph{\AgdaTypewriterFont{}#1}}}
\newcommand\AC[1]{\AgdaInductiveConstructor{#1}}
\newcommand\AF[1]{\AgdaFunction{#1}}
\newcommand\AB[1]{\AgdaBound{#1}}
\renewcommand\AgdaMacro[1]{\AC{#1}}
\usepackage{agda-latex-macros}

% Haskell
\usepackage{minted}
\newcommand\hs[1]{\mintinline{haskell}{#1}}
%% \setminted[haskell]{fontsize=\footnotesize}
%% \usemintedstyle{friendly}

% Math fonts
\usepackage{unicode-math}
\setsansfont{XITSMath-Regular.otf}
\setmathfont{XITSMath-Regular.otf}

\newcommand{\N}{\ensuremath{\mathbb{N}}}

\usepackage{stmaryrd}
\usepackage{amsmath}

% hide some superscripts
\usepackage{newunicodechar}
\usepackage{unicode-math}
\newunicodechar{ᶠ}{\ensuremath{}}
\newunicodechar{ᵐ}{\ensuremath{}}
\newunicodechar{ᵍ}{\ensuremath{}}
\newunicodechar{ᵘ}{\ensuremath{}}
\newunicodechar{ˢ}{\ensuremath{}}
\newunicodechar{ʳ}{\ensuremath{}}
\newunicodechar{ᶜ}{\ensuremath{}}
\newunicodechar{ˡ}{\ensuremath{}}

\newunicodechar{⦉}{\ensuremath{⦇}}
\newunicodechar{⦊}{\ensuremath{⦈}}

%% \usepackage{amsxtra}
%% \usepackage{newunicodechar}
%% \newunicodechar{∷}{::}

%%%%%%%%%%%%%%%%%%%%%%%%%%%%%%%%%%%%%%%%%%
%% Macros
%%%%%%%%%%%%%%%%%%%%%%%%%%%%%%%%%%%%%%%%%%
\renewcommand\alert[1]{\textcolor{mLightBrown}{#1}}
\newcommand\todo[1]{\textcolor{red}{#1}}

%%
%% Functions
%%
\newcommand{\txouts}[1]{\fun{txouts}~ \var{#1}}
\newcommand{\txid}[1]{\fun{txid}~ \var{#1}}
\newcommand{\outs}[1]{\fun{outs}~ \var{#1}}
\newcommand{\balance}[1]{\fun{balance}~ \var{#1}}
\newcommand{\txfee}[1]{\fun{txfee}~ \var{#1}}

\newtheorem{property}{Property}[section]

%% -- DEFINITIONS -----------------------------------------------------------
%% Set the desired typeface of defined terms here and, at the
%% first occurrence of such a term, enclose it in `\defn{...}`.
\newcommand{\defn}[1]{\textit{#1}}   %  defined terms are typeset in italics
%% \newcommand{\defn}[1]{\textbf{#1}}   %  defined terms are typeset in bold
\newcommand\fun[1]{\textbf{#1}}
\newcommand\var[1]{\textit{#1}}

\newcommand\bpar[1]{{\vspace{2pt}\noindent\textbf{#1.}}\enspace}
\newcommand\textbftt[1]{\textbf{\texttt{#1}}}
\newcommand\tsc[1]{\textsc{#1}} % rules, etc...
\newcommand\ie{\textit{i.e.,} }
\newcommand\eg{\textit{e.g.,} }
\newcommand\cf{\textit{c.f.,} }
\newcommand\etal{\textit{et al.}}
\newcommand\etc{\textit{etc.}}
\newcommand\aka{\textit{a.k.a.} }

\usepackage{longtable}

\title{%
Formal specification of the Cardano ledger,\\
mechanized in Agda}

\author{\footnotesize Andre Knispel, \alert{Orestis Melkonian}, James Chapman,
Alasdair Hill, Joosep Jääger, William DeMeo, Ulf Norell\\}
\date{7 April 2024, FMBC, Luxembourg}

\begin{document}

% ** Agda setup
\AgdaNoSpaceAroundCode{}

\setbeamerfont{title}{size=\LARGE}
\setbeamerfont{subtitle}{size=\Large}

%% \titlegraphic{}

\begin{center}
\maketitle
\end{center}

\include{latex/Ledger/Introduction}
\include{latex/Ledger/Crypto}
\include{latex/Ledger/Address}
\include{latex/Ledger/PParams}
\include{latex/Ledger/Chain}
\include{latex/Ledger/Ledger}
\include{latex/Ledger/Transaction}
\include{latex/Ledger/Utxo}
\include{latex/Ledger/Utxo/Properties}
\include{latex/Ledger/Utxow}
\include{latex/Ledger/Utxow/Properties}
\include{latex/Haskell}

\section{Introduction}
\begin{frame}{Motivation}
Much (formally-verified) meta-theoretical work on EUTxO
  \begin{itemize}
  \item ...but all on simplified and idealized settings
  \item ...none cover the full feature set of a realistic blockchain like Cardano
  \end{itemize}
\pause
Previous iterations of Cardano's ledger specification written informally on paper
  \begin{itemize}
  \item Lack the rigor of a mechanized formal artifact
  \item Are not executable and thus require a separate prototype to be implemented
  \end{itemize}
\end{frame}
\begin{frame}{Methodology}
Separation of Concerns
\begin{itemize}
\item Networking: deals with sending messages across the internet
\item Consensus: establishes a common order of valid blocks
\item \alert{Ledger}: decides whether a sequence of blocks is valid
\end{itemize}

\pause
Stay as close to the previous (LaTeX) specification as possible
\begin{itemize}
\item Use of \alert{set theory}
\item System evolution formulated as \alert{state machines}
\end{itemize}

\pause
Use the Agda proof assistant to produce a
\alert{readable} mechanized specification.
\end{frame}

\section{Agda Preliminaries}

\begin{frame}{Axiomatic Set theory}
Easy to define \alert{common operations}:
\setTheory
\pause
\vfill
Finite \alert{maps} as set of tuples:
\setTheoryMaps
\end{frame}

\begin{frame}{Type of transitions}
\begin{center}
\[ Γ ⊢ s \xrightarrow[X]{b} s' \]
\end{center}
\pause
\vfill
\transitionType
\end{frame}

\begin{frame}{Depicting transitions: Triptychs}
\begin{minipage}{.4\textwidth}
\emph{Environments}

\emph{(Signals)}
\end{minipage}
\hfill\vrule\hfill
\begin{minipage}{.4\textwidth}
\emph{States}
\end{minipage}
\hrule
\centering
\vspace{2pt}
\emph{Possible transitions}
\end{frame}


\begin{frame}{Sequencing transitions: Reflexive-transitive closure}
\rtClosure
\end{frame}

\section{Formalization}

\begin{frame}{Basic entities / assumptions}
\begin{itemize}
\item Crypto:
\crypto
\pause
\vspace{-.2cm}
\item Addresses:
\addresses
\pause
\vspace{-.2cm}
\item Tunable parameters:
\parameters
\end{itemize}
\end{frame}

\tikzset{
  every matrix/.style =
  { ampersand replacement = \&,
    matrix of nodes,
    nodes in empty cells },
  every node/.style =
  { minimum height = .5cm
  },
  box/.style =
  { draw, rectangle,
    rounded corners = 1mm,
    minimum width   = 2cm },
  to/.style = {->, thick}
}

\begin{frame}[fragile]{The hierarchy of transitions}
\begin{center}
\begin{tikzpicture}
  \matrix (mat) [row sep = 1.2cm, every node/.style = {align=center}, font=\Large] {
    \&
    \node (chain) {CHAIN}; \\
    \node (newepoch) {NEWEPOCH}; \& \&
    \node (ledger) {LEDGER*}; \\
    \node (ratify) {RATIFY*}; \&
    \node (cert) {CERT*}; \&
    \node (gov) {GOV*}; \&
    \node (utxow) {UTXOW}; \\
    \node (enact) {ENACT}; \& \& \&
    \node (utxo) {UTXO}; \\
  };
  \path
  (chain) edge[to] (ledger)
  (ledger) edge[to] (utxow)
  (ledger) edge[to] (cert)
  (ledger) edge[to] (gov)
  (utxow) edge[to] (utxo)

  (chain) edge[to] (newepoch)
  (newepoch) edge[to] (ratify)
  (ratify) edge[to] (enact)
  ;
\end{tikzpicture}
\end{center}
\end{frame}

\begin{frame}[fragile]{The hierarchy of transitions}
\begin{center}
\begin{tikzpicture}
  \matrix (mat) [row sep = 1.2cm, every node/.style = {align=center}, font=\Large] {
    \&
    \node[box] (chain) {CHAIN}; \\
    \node (newepoch) {NEWEPOCH}; \& \&
    \node[box] (ledger) {LEDGER*}; \\
    \node (ratify) {RATIFY*}; \&
    \node (cert) {CERT*}; \&
    \node (gov) {GOV*}; \&
    \node[box] (utxow) {UTXOW}; \\
    \node (enact) {ENACT}; \& \& \&
    \node[box] (utxo) {UTXO}; \\
  };
  \path
  (chain) edge[to] (ledger)
  (ledger) edge[to] (utxow)
  (ledger) edge[to] (cert)
  (ledger) edge[to] (gov)
  (utxow) edge[to] (utxo)

  (chain) edge[to] (newepoch)
  (newepoch) edge[to] (ratify)
  (ratify) edge[to] (enact)
  ;
\end{tikzpicture}
\end{center}
\end{frame}

\begin{frame}{CHAIN: processing blocks}
\chain
\end{frame}

\begin{frame}{LEDGER: processing transactions}
\scalebox{.9}{\vbox{%
\ledger
}}
\end{frame}

\begin{frame}{LEDGER: The transaction type}
\txType
\end{frame}

\begin{frame}{UTXOW: witnesses \& scripts}
\utxow
\end{frame}

\begin{frame}{UTXO: the ``core'' transition}
\utxo
\end{frame}

\begin{frame}{UTXO: the property of Value Preservation}
\begin{property}[\textbf{Value preservation}]
\pov
\end{property}
\end{frame}

\section{Compiling to executable Haskell}

\begin{frame}{Proving transitions are \alert{computational}}
\computational
\end{frame}

\begin{frame}{Example: \alert{compiling} the UTXOW transition (\alert{Agda} source)}
\scalebox{.8}{\vbox{%
\utxowComp
\vspace{.2cm}\hrule\vspace{.2cm}
\utxowStep
}}
\end{frame}

\begin{frame}[fragile]{Example: \alert{running} the UTXOW transition (\alert{Haskell} target)}
\begin{minted}[fontsize=\footnotesize]{haskell}
import Lib (utxowStep)

utxowSteps :: UTxOEnv -> UTxOState -> [Tx] -> Maybe UTxOState
utxowSteps = foldM . utxowStep
\end{minted}
\end{frame}
\begin{frame}[fragile]{Example: \alert{running} the UTXOW transition (\alert{Haskell} target)}
\begin{minted}[fontsize=\scriptsize]{haskell}
spec :: Spec
spec = describe "utxowSteps" $ it "results in the expected state" $
  utxowSteps initEnv initState [testTx1, testTx2] @?= Just (MkUTxOState
    { utxo = [ (1,0) .-> (a0, (890, Nothing))
             , (2,0) .-> (a2, (10,  Nothing))
             , (2,1) .-> (a1, (80,  Nothing)) ]
    , fees = 20 })
  where
  testTx1, testTx2 :: Tx

  initEnv :: UTxOEnv
  initEnv = MkUTxOEnv {slot = 0, pparams = ...}

  initUTxO :: UTxO
  initUTxO = [ (0, 0) .-> (a0, (1000, Nothing)) ]

  initState :: UTxOState
  initState = MkUTxOState {utxo = initUTxO, fees = 0}
\end{minted}
\end{frame}

\section{Conclusion}
\begin{frame}{Future Work}
Compilation issues:
  \begin{itemize}
  \item Automate away \alert{boilerplate}
  \item Finalize \alert{conformance-testing} integration
  \item Randomly test (proven) Agda statements by translating
    to \alert{Quickcheck} properties
  \item Optimizations in implementation $\to$ \alert{refinements} in formalization
  \end{itemize}

\pause
Expand the scope of the formalization
  \begin{itemize}
  \item Prove more interesting meta-theoretical \alert{properties}
  \item Cover previous \alert{eras}: ``keeping up with the past''
  \item Towards verifying \alert{smart contracts}
  \end{itemize}
\end{frame}
\begin{frame}[standout]
Questions?
\vfill
\begin{center}
\alert{\url{https://intersectmbo.github.io/formal-ledger-specifications/}}
\end{center}
\end{frame}

\end{document}
