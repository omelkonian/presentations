%% \documentclass[aspectratio=43]{beamer}
\documentclass[aspectratio=169]{beamer}
\usetheme[
  block=fill,
  background=light,
  titleformat=smallcaps,
  progressbar=frametitle,
  numbering=none,
]{metropolis}
%% \setbeamersize{text margin left=.5cm,text margin right=.5cm}
\usepackage{appendixnumberbeamer}

%%%%%%%%%%%%%%%%%%%%%%%%%%%%%%%%%%%%%%%%%%
%% Packages
%%%%%%%%%%%%%%%%%%%%%%%%%%%%%%%%%%%%%%%%%%

%% \usepackage{subfiles} % for modular TeX development
\usepackage{xargs} % multiple optional arguments
\usepackage{xspace} % no-argument aliases
\usepackage{environ} % flexible environments

% Tikz
\usepackage{tikz}
\usetikzlibrary{arrows,positioning,matrix,fit,backgrounds,decorations.text,decorations.pathmorphing,calc,shapes}

% Colors
\usepackage{xcolor}
\usepackage{contour}

% Images
\usepackage{graphics}
\graphicspath{{img/}}

% Quotes
\usepackage{csquotes}
\SetBlockThreshold{2}
\newcommand\myblockquote[2]{%
  \blockquote{\hspace*{2em}\emph{`#1'}#2}\par}

% Agda
\usepackage{agda}
%% \renewcommand\\[1][XXX]{\newline} % fix `\\[\AgdaEmptyExtraSkip]%` occurrences
\usepackage{fontspec}
\newcommand\agdaFont{StrippedJuliaMono}
\newcommand\agdaFontOptions{
Path=fonts/,
Extension=.ttf,
UprightFont=*-Regular,
BoldFont=*-Bold,
ItalicFont=*-RegularItalic,
BoldItalicFont=*-BoldItalic,
%% Scale=MatchLowercase
Scale=0.80
%% Scale=MatchUppercase
}
\newfontfamily{\AgdaSerifFont}{\agdaFont}[\agdaFontOptions]
\newfontfamily{\AgdaSansSerifFont}{\agdaFont}[\agdaFontOptions]
\newfontfamily{\AgdaTypewriterFont}{\agdaFont}[\agdaFontOptions]
\renewcommand{\AgdaFontStyle}[1]{{\AgdaSansSerifFont{}#1}}
\renewcommand{\AgdaKeywordFontStyle}[1]{{\AgdaSansSerifFont{}#1}}
\renewcommand{\AgdaStringFontStyle}[1]{{\AgdaTypewriterFont{}#1}}
\renewcommand{\AgdaCommentFontStyle}[1]{{\AgdaTypewriterFont{}#1}}
\renewcommand{\AgdaBoundFontStyle}[1]{{\emph{\AgdaTypewriterFont{}#1}}}
\newcommand\AC[1]{\AgdaInductiveConstructor{#1}}
\newcommand\AF[1]{\AgdaFunction{#1}}
\newcommand\AB[1]{\AgdaBound{#1}}

% Math fonts
\usepackage{unicode-math}
\setsansfont{XITSMath-Regular.otf}
\setmathfont{XITSMath-Regular.otf}

\newcommand{\N}{\ensuremath{\mathbb{N}}}

\usepackage{stmaryrd}
\usepackage{amsmath}

% hide some superscripts
\usepackage{newunicodechar}
\usepackage{unicode-math}
\newunicodechar{ᶠ}{\ensuremath{}}
\newunicodechar{ᵐ}{\ensuremath{}}
\newunicodechar{ᵍ}{\ensuremath{}}
\newunicodechar{ᵘ}{\ensuremath{}}
\newunicodechar{ˢ}{\ensuremath{}}
\newunicodechar{ʳ}{\ensuremath{}}
\newunicodechar{ᶜ}{\ensuremath{}}
\newunicodechar{ˡ}{\ensuremath{}}

\newunicodechar{⦉}{\ensuremath{⦇}}
\newunicodechar{⦊}{\ensuremath{⦈}}

%% \usepackage{amsxtra}
%% \usepackage{newunicodechar}
%% \newunicodechar{∷}{::}

%%%%%%%%%%%%%%%%%%%%%%%%%%%%%%%%%%%%%%%%%%
%% Macros
%%%%%%%%%%%%%%%%%%%%%%%%%%%%%%%%%%%%%%%%%%
\renewcommand\alert[1]{\textcolor{mLightBrown}{#1}}
\newcommand\todo[1]{\textcolor{red}{#1}}

%%
%% Functions
%%
\newcommand{\txouts}[1]{\fun{txouts}~ \var{#1}}
\newcommand{\txid}[1]{\fun{txid}~ \var{#1}}
\newcommand{\outs}[1]{\fun{outs}~ \var{#1}}
\newcommand{\balance}[1]{\fun{balance}~ \var{#1}}
\newcommand{\txfee}[1]{\fun{txfee}~ \var{#1}}

\newtheorem{property}{Property}[section]

%% -- DEFINITIONS -----------------------------------------------------------
%% Set the desired typeface of defined terms here and, at the
%% first occurrence of such a term, enclose it in `\defn{...}`.
\newcommand{\defn}[1]{\textit{#1}}   %  defined terms are typeset in italics
%% \newcommand{\defn}[1]{\textbf{#1}}   %  defined terms are typeset in bold
\newcommand\fun[1]{\textbf{#1}}
\newcommand\var[1]{\textit{#1}}

\newcommand\bpar[1]{{\vspace{2pt}\noindent\textbf{#1.}}\enspace}
\newcommand\textbftt[1]{\textbf{\texttt{#1}}}
\newcommand\tsc[1]{\textsc{#1}} % rules, etc...
\newcommand\ie{\textit{i.e.,} }
\newcommand\eg{\textit{e.g.,} }
\newcommand\cf{\textit{c.f.,} }
\newcommand\etal{\textit{et al.}}
\newcommand\etc{\textit{etc.}}
\newcommand\aka{\textit{a.k.a.} }

\usepackage{longtable}

\title{%
Formal specification of the Cardano ledger,
mechanized in Agda}

\author{Andre Knispel, \alert{Orestis Melkonian}, James Chapman,
Alasdair Hill, Joosep Jääger, William DeMeo, Ulf Norell}
\date{21 March 2024, FM meeting \@ IOG}

\begin{document}

% ** Agda setup
\AgdaNoSpaceAroundCode{}

\setbeamerfont{title}{size=\LARGE}
\setbeamerfont{subtitle}{size=\Large}

%% \titlegraphic{}

\begin{center}
\maketitle
\end{center}

\include{latex/Ledger/Introduction}

\begin{frame}{Intro}
\myblockquote{\textit{
Some quotes are worth more than others.
}}{\\---someone}
\end{frame}

\begin{frame}{Motivation}
\begin{itemize}
\item Explore another \alert{point in the design space}
\item Provide a \alert{constructive} perspective on nominal techniques
\item Do this \alert{without changing the system itself} --- as an Agda library
\item Make it \alert{ergonomic} for the user to use the library as a tool for dealing with names
(e.g. working on some syntax with binding)
\item Mechanise existing (but also new?) \alert{meta-theoretical results}
\end{itemize}
\end{frame}

\section{Agda Preliminaries}

\begin{frame}{Separation of concerns}
\begin{itemize}
\item \textbf{Networking}: deals with sending messages across the internet.
\item \textbf{Consensus}: establishes a common order of valid blocks.
\item \textbf{Ledger}: decides whether a sequence of blocks is valid.
\end{itemize}
\end{frame}

\begin{frame}{State transitions}
\begin{center}
\[ Γ ⊢ s \xrightarrow[X]{b} s' \]
\end{center}
\vfill
\transitionType{}
\end{frame}

\begin{frame}{Triptychs}
\begin{minipage}{.4\textwidth}
\emph{Environments}

\emph{(Signals)}
\end{minipage}
\hfill\vrule\hfill
\begin{minipage}{.4\textwidth}
\emph{States}
\end{minipage}
\hrule
\centering
\vspace{2pt}
\emph{Possible transitions}
\end{frame}

\begin{frame}{Reflexive-transitive closure}
\rtClosure{}
\end{frame}

\begin{frame}{Set theory}
\setTheory{}
\end{frame}


\begin{frame}{Future Work}
\begin{itemize}
\item More meta-programming automation to minimise overhead
  \begin{itemize}
  \item corresponding laws and equivariance lemmas follow the same type-directed structure
 as the swap operation itself
  \end{itemize}
\item Another case study on \emph{cut elimination} for first-order logic
  \begin{itemize}
  \item need to work with entities that are not finitely supported
  \item also includes name abstraction over proof trees
  \end{itemize}
\item Formalise the constructive \emph{total} concretion function, which seems novel
\end{itemize}
\end{frame}

\begin{frame}[standout]
Questions?
\vfill
\begin{center}
\alert{\url{https://omelkonian.github.io/nominal-agda}}
\end{center}
\end{frame}

\end{document}
