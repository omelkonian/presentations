\documentclass{beamer}
\usetheme[
  block=fill,
  background=light,
  titleformat=smallcaps,
  progressbar=frametitle,
  numbering=none,
]{metropolis}

% Tikz
\usepackage{tikz}
\usetikzlibrary{arrows,fit,backgrounds,positioning,calc,trees,decorations.pathreplacing}
%\usetikzlibrary{arrows,mindmap,trees,fit,backgrounds,}
\usepackage{environ}
\makeatletter
\newsavebox{\measure@tikzpicture}
\NewEnviron{scaletikzpicturetowidth}[1]{%
  \def\tikz@width{#1}%
  \def\tikzscale{1}\begin{lrbox}{\measure@tikzpicture}%
  \BODY
  \end{lrbox}%
  \pgfmathparse{#1/\wd\measure@tikzpicture}%
  \edef\tikzscale{\pgfmathresult}%
  \BODY
}
\makeatother

% Fonts
\usepackage{fontspec}
\setmainfont{LinLibertine}[
  Extension = .otf,
  Path = ./fonts/,
  UprightFont = *_R,
  BoldFont = *_RB,
  ItalicFont = *_RI
]
\setsansfont{LinBiolinum}[
  Extension = .otf,
  Path = ./fonts/,
  UprightFont = *_R,
  BoldFont = *_RB,
  ItalicFont = *_RI
]

% Colors
\usepackage{xcolor,color,colortbl}
\definecolor{Gray}{rgb}{0.9,0.9,0.9}
\definecolor{mybl}{HTML}{1D577A}
\definecolor{myrd}{HTML}{C03425}
\definecolor{mygr}{HTML}{4F932F}
\definecolor{mypr}{HTML}{6B1B7F}
\definecolor{mybk}{HTML}{000000}
\definecolor{myye}{HTML}{F9F668}
\definecolor{num}{HTML}{FF7506}

% Images
\usepackage{graphics}
\graphicspath{{figures/}}

% Code
\usepackage{minted}
\usemintedstyle{friendly}
%\usemintedstyle{tango}

% Extra
\usepackage{adjustbox}
\usepackage{multicol}

% Checklist
\usepackage{ifthen}
\usepackage{pifont}
\newcommand{\done}{\ding{51}}

% Macros
\renewcommand\alert[1]{\textcolor{mLightBrown}{#1}}

\newcommand{\framepic}[1]{{\usebackgroundtemplate{%
  \includegraphics[width=\paperwidth,height=\paperheight]{#1}}
	\begin{frame}{}\end{frame}
}}

\newcommand{\checklist}[4]{
  \begin{frame}[standout]
	  \begin{itemize}
    \item Reactive \ifnum #1=1 \done \fi
    \item Heterogeneous \ifnum #2=1 \done \fi
    \item Extensible \ifnum #3=1 \done \fi
    \item Abstract \ifnum #4=1 \done \fi
    \end{itemize}
  \end{frame}
}
%----------------------------------------------------------------------------

% Beamer
\title{RHEA}
\subtitle{A Reactive, Heterogeneous, Extensible, and Abstract Framework for Dataflow Programming}
\author{\alert{Orestis Melkonian}, Angelos Charalambidis}
\date{October 29, 2018}
\institute{Utrecht University, National Center for Scientific Research "Demokritos"}
\titlegraphic{\vspace{7cm}\flushright\includegraphics[angle=90,origin=c,scale=0.2]{ufonds}}

\begin{document}
  % Marble Diagrams
\newcommand{\op}[1]{
  	\node[draw,rounded corners,thick,fit={(0,.8)(5,1.4)}] (id) {};
  	\node[anchor=base,inner sep=0pt] at (id.center) {#1};
}
\newcommand{\stream}[1]{
	\draw[->,very thick] (0,#1) -- (5,#1);
}
\newcommand{\dotstream}[2]{
	\draw[->,very thick] (0,#1) -- (#2,#1);
	\draw[dotted] (#2,#1) -- (5,#1);
}
\newcommand{\onnext}[4]{
	\draw (#1,#2) node [draw,circle,fill=#3] {\tiny #4};
}
\newcommand{\error}[2]{
	\draw[myrd, line width=1mm] (#1-0.3,#2-0.3) -- (#1+0.3,#2+0.3) (#1+0.3,#2-0.3) -- (#1-0.3,#2+0.3);
}
\newcommand{\complete}[2]{
	\draw[mygr, line width=1mm] (#1,#2-0.2) -- (#1,#2+0.2);
}
\newcommand\mat[2]{
	\node [rectangle,minimum size=5.5mm,inner sep=0pt,outer sep=0pt,thick,fill=mybr,opacity=.8] at (#1,#2) {};
}
\newcommand\llist[2]{
	\node[draw,rounded corners,thick,fit={(#1-0.1,-0.3)(#2+0.1,0.3)}]{};
}

% Dataflow graphs
\tikzset{
	dataflow/.style={draw, fill=myye!70, rounded corners, thick},
	empty/.style={draw=none, inner sep=.3cm},
  	machine/.style={draw, fill=myrd!80, rounded corners, thick, inner sep=.3cm},
  	task/.style={draw, fill=mygr!80, rounded corners, thick, inner sep=.5cm},
  	to/.style={->,>=stealth',shorten >=1pt,semithick},
  	net/.style={to, red, dotted},
  	opt/.style={->,>=stealth',shorten >=1pt,semithick,draw=red},
  	point/.style={draw, inner sep=0pt, circle, fill=black},
  	every matrix/.style={ampersand replacement=\&, column sep=.7cm, row sep=.35cm},
  	every node/.style={align=center, anchor=base}
}

\newcommand\myimage[2]{
	\begin{figure}
		\centering
	 	\includegraphics[scale=#2]{#1}
	  	\label{fig:#1}
	\end{figure}
}

\newcommand\mydiag[1]{
	\begin{figure}
		\centering
	 	\input{diagrams/#1.tikz}
	  	\label{fig:#1}
	\end{figure}
}

\newcommand\optim[3]{
	\begin{figure}
	  \begin{adjustbox}{width=#1}
	  \input{diagrams/optimization/#3B.tikz}
	  \hspace{-.4cm}
	  \input{diagrams/optimization/rightarrow#2.tikz}
	  \hspace{-.4cm}
	  \input{diagrams/optimization/#3A.tikz}
	  \end{adjustbox}
		\label{fig:#3}
	\end{figure}
}

\newcommand\optimizationS[2]{
	\optim{.5\columnwidth}{#1}{#2}
}

\newcommand\optimization[2]{
	\optim{.7\columnwidth}{#1}{#2}
}

\newcommand\optimizationL[2]{
	\optim{.9\columnwidth}{#1}{#2}
}

\newcommand\codefig[2]{
  \begin{figure}
    \begin{minipage}{.3\columnwidth}
    	  \input{diagrams/#1.tikz}
    \end{minipage}
    \hfill
    \begin{minipage}{.7\columnwidth}
      \jvsf{code/#1.java}
    \end{minipage}
  \end{figure}
}

% Source Code
\usemintedstyle{friendly}
\newcommand{\cd}[1]{\mintinline{java}{#1}}
\newcommand\mintedfs{0.2cm}
\newmintedfile[hs]{hs}{frame=lines, framesep=\mintedfs}
\newmintedfile[hssm]{hs}{fontsize=\footnotesize, frame=lines, framesep=\mintedfs}
\newmintedfile[jv]{java}{frame=lines, framesep=\mintedfs}
\newmintedfile[scala]{scala}{frame=lines, framesep=\mintedfs}
\newmintedfile[scalas]{scala}{fontsize=\footnotesize, frame=lines, framesep=\mintedfs}
\newmintedfile[scalaxs]{scala}{fontsize=\tiny, frame=lines, framesep=\mintedfs}
\newmintedfile[lscalaxs]{scala}{numberblanklines=false, numbers=right, numbersep=0pt, fontsize=\tiny, frame=lines, framesep=\mintedfs}
\newmintedfile[jvsf]{java}{fontsize=\footnotesize, frame=lines, framesep=\mintedfs}
\newmintedfile[jvs]{java}{fontsize=\footnotesize, baselinestretch=0.8}
\newmintedfile[jvxs]{java}{fontsize=\tiny, frame=lines, framesep=\mintedfs}
\newmintedfile[ljvxs]{java}{numberblanklines=false, numbers=right, numbersep=0pt, , fontsize=\tiny, frame=lines, framesep=\mintedfs}
\newmintedfile[cp]{cpp}{fontsize=\tiny}

	\maketitle
	
	\begin{frame}{Robot Operating System (ROS)}
    \begin{itemize}
		\item Most popular middleware for robotic applications
	  \item Provides a Publish-Subscribe messaging platform \vspace{1cm}
	      \mydiag{pubsub}
	  \end{itemize}
	\end{frame}
	
	\begin{frame}{ROS Code}
	  \cp{code/ros.cpp}
	\end{frame}

	\begin{frame}{Dataflow computational model}
		\begin{itemize}
		\item Completely decentralized
		  \begin{itemize}
		  \item Independent nodes communicating with each other
		  \end{itemize}
		\item No control-flow
		\item Implicit concurrency
		\end{itemize}
		\mydiag{nat}
	\end{frame}
	
	\begin{frame}{Motivation: Other Domains}
	  \begin{itemize}
	  \item Big Data
	    \begin{itemize}
	    \item Apache Flink
	    \item Map-Reduce
	    \item RX framework
	    \end{itemize}
    \item Interactive Systems
      \begin{itemize}
      \item UIs (ReactJS)
      \item Games (Yampa)
      \end{itemize}
	  \item Neural Networks (TensorFlow)
	  \end{itemize}
	\end{frame}
	
	\begin{frame}{Motivation: Robotics}
		\begin{itemize}
		\item Robot Perception Architecture (RPA)
		\item Many dataflow examples in control theory
		  \myimage{pid}{0.35}
		\end{itemize}
	\end{frame}
	
	\begin{frame}{RHEA}
    	\myimage{architecture}{0.4}
	  \begin{itemize}
	  \item Dataflow framework for the JVM
	  \item Current frontends only in Java \& Scala
	  \item Set of libraries in \alert{github.com/rhea-flow}
	  \end{itemize}
	\end{frame}
	
	\begin{frame}{Stream Language: Sources and single-input nodes}
	  \begin{minipage}{.5\textwidth}\centering
		  \jvsf{code/singleinput.java}
		\end{minipage}
		\hfill
		\begin{minipage}{.4\textwidth}\centering
		  \begin{tikzpicture}
  \matrix{
    \node[dataflow] (just) {\small just [1, 2, 3]}; \\
    \node[dataflow] (map) {\small map \{$+1$\}};  \\
  };
  \draw[to] (just) -- (map);
\end{tikzpicture}

	  \end{minipage}
  \end{frame}
  \begin{frame}{Stream Language: Multiple-input nodes}
	  \begin{minipage}{.5\textwidth}\centering
		  \jvsf{code/multipleinput.java}
		\end{minipage}
		\hfill
		\begin{minipage}{.4\textwidth}\centering
		  \begin{tikzpicture}
  [every matrix/.append style={column sep=0.05cm}]
  \matrix{
    \node[dataflow] (fst) {\small 1..10};
    \& \& \node[dataflow] (snd) {\small 1..10}; \\
    \& \node[dataflow] (zip) {\small zip \{$+$\}};  \\
  };
  \draw[to] (fst) -- (zip);
  \draw[to] (snd) -- (zip);
\end{tikzpicture}

	  \end{minipage}
	\end{frame}
  \begin{frame}{Stream Language: Split}
	  \begin{minipage}{.5\textwidth}\centering
		  \jvsf{code/split.java}
		\end{minipage}
		\hfill
		\begin{minipage}{.4\textwidth}\centering
		  \input{diagrams/split.tikz}
	  \end{minipage}
	\end{frame}
	\begin{frame}{Stream Language: Cycle}
	  \begin{minipage}{.5\textwidth}\centering
		  \jvsf{code/nat2.java}
		\end{minipage}
		\hfill
		\begin{minipage}{.4\textwidth}\centering
		  \input{diagrams/nat2.tikz}
	  \end{minipage}
	\end{frame}
	\begin{frame}{Stream Language: Actions}
	  \begin{minipage}{.5\textwidth}\centering
		  \jvsf{code/nat3.java}
		\end{minipage}
		\hspace{1cm}
		\begin{minipage}{.3\textwidth}\centering
		  \input{diagrams/nat3.tikz}
	  \end{minipage}
	\end{frame}
	
	\begin{frame}{Application: Robot Panel}
	  \mydiag{control_panel}
	\end{frame}
	
	\begin{frame}[standout]
	  Robot panel demo
	\end{frame}
	
	\begin{frame}{Application: Robot Hospital Guide}
	  \begin{enumerate}
	  \item Robot guides patients to parts of the hospital
	  \item Patient holds a smartphone that broadcasts bluetooth signals
	  \item Robot adjusts its speed, according to distance
	  \end{enumerate}
	\end{frame}
	\begin{frame}{Application: Robot Hospital Guide}
	  \mydiag{hospital}
	\end{frame}
	\begin{frame}{Application: Robot Hospital Guide}
	  \jvs{code/hospital.java}
	\end{frame}
	
	\begin{frame}{RHEA as a coordination language}
	  \begin{itemize}
	  \item Declarative glue code
	  \item Multiple heterogeneous devices/streams
	  \item Dataflow in the large, whatever in the small
	  \end{itemize}
	\end{frame}
	
	\begin{frame}{Optimizations}
	  Series of semantics-preserving graph transformations
	  \begin{itemize}
	  \item Proactive filtering
	  \item Granularity adjustment
	  \end{itemize}
	\end{frame}
	
	\begin{frame}{Optimizations: Proactive Filtering}
		Transfer as few elements as possible
		\optimization{1}{maptake}
	  \optimization{1}{mapfilter}
	  \optimization{3}{concatdistinct}
	\end{frame}
	
	\begin{frame}{Optimizations: Granularity Adjustment}
	  Merge nodes
		\optimizationS{1}{mergemap}
	  \optimizationS{1}{combfilterexists}
	  \optimizationS{1}{combmapexists}
	  \optimizationS{2}{combmapzip}
	\end{frame}
	
	\begin{frame}{Distribution: Task Fusion}
    \begin{enumerate}
    \item If desired granularity not reached, perform task fusion:
      \optimizationL{1}{fusion}
    \end{enumerate}	
   \end{frame}
    
  \begin{frame}{Distribution: Node Placement}
	  \begin{enumerate} \setcounter{enumi}{1}
	  \item Place nodes in the available machines, in order to:
		  \begin{itemize}
		  \item minimize communication overhead
		  \item satisfy hard constraints (e.g. ROS not available on raspberry)
		  \end{itemize}
	  \optimizationL{2}{partition}
	  \end{enumerate}
	\end{frame}
	
	\begin{frame}{Distribution: Serialization}
	  \begin{enumerate} \setcounter{enumi}{2}
	  \item Streams can terminate either with \cd{Complete} or \cd{Error}. 
	    \begin{itemize}
	    \item Necessary to materialize them when transferring
      \end{itemize}
	  \mydiag{serialization}
	  \end{enumerate}
	\end{frame}
  
  \begin{frame}{Limitations}
    \begin{itemize}
    \item Difficult to extend the available operators
    \item The surface syntax is not strict enough
      \begin{itemize}
      \item Only single-assignment in \cd{Stream} variables
      \item Specific program structure (configuration $\rightarrow$ dataflow)
      \item Only pure functions as arguments to higher-order operators
      \end{itemize}
    \end{itemize}
  \end{frame}
  
  \begin{frame}{Future Work}
    \begin{itemize}
    \item More sophisticated optimizations
    \item Reinforcement learning for node placement
    \item Dynamic reconfiguration (hot-swapping code)
    \item Erlang-style error handling
    \item Machine-learning backend
    \item $\dots$
    \end{itemize}
  \end{frame}
  
  \begin{frame}{Conclusion}
    \begin{itemize}
    \item Some domains are still full of low-level techniques
    \item The FP paradigm can overcome this quite nicely
    \item Higher, higher, higher!
    \end{itemize}
  \end{frame}
  
  \begin{frame}[standout]
    Questions?
  \end{frame}
  
  	
\end{document}
